De controller moet in staat om een servomotor over 180\textdegree  aan te sturen met \gls{pwm} signalen tussen 1,25 ms en 1,75 ms. De reset verloopt synchroon met de klok, dit is om te verhinderen dat het \gls{pwm} signaal vervormt zou worden door de slechte timing van een reset. In het geval van asynchrone reset, zou het  signaal onbedoeld langer kunnen worden dan de maximum pulsduur (1,75 ms).\\
\\
De controller werkt met een klok van 50 Hz en een servoklok van 510 kHz. Hiermee rekening houdende blijkt dat de minimum pulsduur van 1,25 ms overeen komt met 637,5 servoklok ticks, het midden van 1.5 ms komt overeen met 765 servoklok ticks en de maximum pulsduur van 1,75 ms komt overeen met 892,5 servoklok ticks. In \gls{vhdl} wordt gewerkt met unsigned variabelen, dus 637,5 en 892,5 worden afgerond naar respectievelijk 637 en 892.\\
\\
Elke servocontroller heeft een adres. Als er een positie gestuurd wordt die vooraf gegaan wordt door het correcte adres, dan moet de controller deze instructie uitvoeren. Indien een ander adres vermeld wordt dan moet de controller de opdracht negeren.\\
Ook is er een \textit{broadcast} adres waar alle controllers  naar moeten luisteren, namelijk 255. Als een signaal uitgestuurd wordt met het \textit{broadcast} adres dan moeten alle aangesloten controllers hetzelfde \gls{pwm} signaal generen.\\
\\
Onder normale operatie zal de controller de positie behouden, maar als er iets fout loopt zal de controller de servomotor terugzetten naar zijn neutrale positie. De neutrale positie wordt hier gedefinieerd als op 90\textdegree, en komt overeen met een pulsduur van 1,5ms. De controller keert ook naar deze neutrale positie terug indien een reset opgeroepen wordt.\\
\\
Het Done signaal werk met \textit{tri-state logic}, dit betekend dat het op een bus aangesloten kan worden. Hierdoor kunnen meerdere controllers gebruik maken van \'{e}\'{e}nzelfde bus, waardoor er bespaard kan worden op het aantal kabels dat nodig is.\\
\\
Als de aansturende microcontroller tijdens het aansturen van de controller beslist om een taak van hogere prioriteit uit te voeren, moet de controller dit kunnen afhandelen en naar de neutrale positie gaan.\\
